\documentclass[11pt, a4paper]{article}
\usepackage{CJKutf8}
\usepackage{amsthm}
\usepackage{ulem}
\usepackage{xcolor}
\usepackage{amsmath}
\usepackage{amssymb}
\usepackage{courier}
\usepackage{geometry}
\usepackage{enumitem}
\usepackage{graphicx}
\usepackage{listings}
\usepackage{algorithm}
\usepackage{algorithmic}
\usepackage{indentfirst}
%\usepackage{float}
\usepackage[perpage,stable]{footmisc} 
\geometry{left=2.7cm, right=2.7cm, top=3cm, bottom=3cm}


\usepackage{graphics}

\usepackage[colorlinks,linkcolor=black,anchorcolor=blue,citecolor=green,
  %	CJKbookmarks=true,
]{hyperref}
\hypersetup{unicode}

\linespread{1.4}
\usepackage{indentfirst}
\setlength{\parindent}{2em}
\hypersetup{hidelinks}

\usepackage{listings}
\lstset{
  numbers=left,
  texcl=true,
  escapechar=`,
  backgroundcolor=\color{lightgray!40!white}, 
  commentstyle=\rm\color{green!30!black},
  basicstyle=\footnotesize\tt,        % the size of the fonts that are used for the code
  breakatwhitespace=false,         % sets if automatic breaks should only happen at whitespace
  breaklines=true,                 % sets automatic line breaking
  captionpos=b,                    % sets the caption-position to bottom
  extendedchars=false,              % lets you use non-ASCII characters; for 8-bits encodings only, does not work with UTF-8
  frame=single,                    % adds a frame around the code
  language=C++,                 % the language of the code
  keywordstyle=\color{blue!70}\bfseries,
  showspaces=false,                % show spaces everywhere adding particular underscores; it overrides 'showstringspaces'
  showstringspaces=false,          % underline spaces within strings only
  showtabs=false,                  % show tabs within strings adding particular underscores
  tabsize=2                       % sets default tabsize to 2 spaces
}

\begin{document}
\begin{CJK*}{UTF8}{gbsn}
  \title{\bf 2016夏季学期C++课程大作业报告}
  \author{马玉坤-1150310618}
  \date{2016年7月30日}
  \maketitle
  \renewcommand{\contentsname}{\textbf{目录}}
  \tableofcontents
  \newpage
  \newpage

  \section{前言}

  五门夏季选修课加其他的琐事,现在才写完大作业。非常感谢又萌又强的三位助教,让我在C++课上甚至课下都学到了很多东西。尽管之前用C++写过不少代码,但是就编程知识,特别是面向对象的思想和方法来讲,这三个星期学到的新东西非常非常多。另外,由于代码逻辑很直(一点都不绕),所以加的注释并不多(其实是因为懒),求助教谅解。


  \section{主要设计}

  \subsection{必须要说的}

  为了节约空间(删除一大堆大小为1M的文件后,添加一大堆2M的文件时,无法充分利用之前删除的文件所占的数据空间),我仿照文件系统中的“单元空间”,将数据区域分成若干个64KB大小的“格子”。每个“格子”中,有8B的空间存储下一个“格子”的位置,这样就形成了一个链表。从数据区域读某个文件时,实际上就是将一个链表读入内存。写文件类似。同时,setHeader还要记录空白区域的首地址,这样当添加文件时也更为方便。

  其余的实现与段艺学长的指导书中的实现大致相同。


  \subsection{FileSet}
  
  \subsubsection{属性}
  
  \begin{lstlisting}
    // 归档文件的文件指针
    FILE* archive;

    // 指向SetHeader的指针
    SetHeader* header;

    // 存储FileTag数组
    FileTag** fileTags;

    // 支持的最大文件数量
    unsigned int maxFileNumber;
  \end{lstlisting}

  \subsubsection{方法}
  \begin{lstlisting}

    // 指定大小和位置,将data写入归档文件
    inline void writeToArchive(byte* data, unsigned int len, long long pos);

    // 将data写入归档文件的文件区中空闲的区域
    inline long long writeFileToArchive(byte* data, unsigned int len);

    // 从归档文件的文件区中删除起始位置为pos的文件
    inline void deleteFileFromArchive(long long pos);

    // 从归档文件的文件区中读入起始位置为pos,大小为len的文件
    inline byte* readFileFromArchive(long long pos, unsigned int len);

    // 计算整个checkSum的md5值
    inline byte* calcCheckSum();

    // 添加文件
    bool addFile(char* filePath);

    // 删除文件
    bool deleteFile(char* fileName);

    // 获取文件
    bool fetchFile(char* fileName, char* filePath);

    // 打印文件
    void printFileList(); 

  \end{lstlisting}

  \subsection{SetHeader}

  \subsubsection{属性}

  \begin{lstlisting}
    
    // 文件的md5值
    byte checkSum[16];

    // 最大文件数量
    unsigned int maxFileNumber;

    // 当前文件数量
    unsigned int currentFileNumber;

    // 数据区域中空白区域的首地址
    long long emptyPosition;

    // 文件标记
    char setMark[9];
  \end{lstlisting}

  \subsubsection{方法}

  \begin{lstlisting}
    // 返回一个SetHeader序列化为unsigned char数组后的结果
    byte* toBytes();
  \end{lstlisting}

  \subsection{FileBlock}

  FileBlock的作用是存储数据区域中一个“小格”表示的链表结点。

  \subsubsection{属性}

  \begin{lstlisting}
    // 所在链表中下一个“小格”的位置
    long long nextPosition;

    // 当前小格中,存储的数据的大小,最大为64 * 1024 - 12
    unsigned int size;

    // 当前小格存储的数据
    byte data[64*1024 - 12];
  \end{lstlisting}

  \subsubsection{方法}

  \begin{lstlisting}
    // 返回将一个FileBlock对象序列化为byte数组后的结果
    byte* toBytes();
  \end{lstlisting}

  \subsection{FileTag}

  \subsubsection{属性}
  
  \begin{lstlisting}
    // 小文件数据链表中第一项的地址
    long long filePosition;

    // 小文件的大小
    unsigned int fileSize;

    // 当前小文件是否被删除(0代表未删除)
    byte deleted;

    // 文件名
    char fileName[256];
  \end{lstlisting}

  \subsubsection{方法}
  
  \begin{lstlisting}
    // 返回将一个FileTag对象序列化后的结果
    byte* toBytes();
  \end{lstlisting}

  \newpage

\end{CJK*}
\end{document}
